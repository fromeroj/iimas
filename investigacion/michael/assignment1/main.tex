\documentclass{article}
\usepackage{listings}
\usepackage[utf8]{inputenc}
\usepackage{ amssymb }
\usepackage{amsfonts}
\usepackage{cite}
\usepackage{graphicx}
\usepackage{amsthm}
\usepackage[table]{xcolor}

% Theorem Styles
\newtheorem{theorem}{Theorem}[section]
\newtheorem{lemma}[theorem]{Lemma}
\newtheorem{proposition}[theorem]{Proposition}
\newtheorem{corollary}[theorem]{Corollary}
% Definition Styles
\theoremstyle{definition}
\newtheorem{definition}{Definition}[section]
\newtheorem{example}{Example}[section]
\theoremstyle{remark}
\newtheorem{remark}{Remark}

\title{Sperner Lema, from a kripke model's perspective}
%\author{Fabian Romero}
\date{}
\begin{document}
\maketitle

\abstract{This is a preeliminary effort to describe Sperner lemma}

\section{One dimentional Sperner's lemma }

On the case of dimention one the weak Sperner lemma can be stated as follows:

\begin{theorem}[weak one dimentional Sperner's lemma]
    Given an interval $u$ subdivided into $k-1$ subintervals, and the $k$ interval endpoints labeled with either $p$ or $q$, if the boundary points of $u$ have different label, there is one interval having different labels on its endpoints.\\
\end{theorem}

The standard Sperner lemma is almost identical, changing:
``there is one interval'' for ``there is a odd number of intervals''.\\





Sperner's lema is a fixed point theorem.

Sperner's lemma states that every Sperner coloring (described below) of a triangulation of an n-dimensional simplex contains a cell colored with a complete set of colors

asdlaskdj asdk  \cite{Huang_onthe} sdfsdf


\section{Conceptos Básicos}
\begin{enumerate}
  
\item[\bf{Problema 1}] Define claramente que es un sistema complejo.\\
\item[\bf{Respuesta}] Los sistemas complejos son sistemas que exhiben varias características  Incluyendo: \\
\begin{itemize}
\item{\bf Ciclos de retroalimentación} Donde los cambios de una variable resultan en una amplificación (retroalimentación positiva) ó una amortiguación (retroalimentación negativa).
\item{\bf Interdependencia}  Muchas variables fuertemente interdependientes, con múltiples entradas contribuyendo a las salidas observadas. 
\item{\bf Caos, Autosemejanza} Extrema sensibilidad a las condiciones iniciales, geometría fractal y criticidad auto-organizada. 
\item{\bf Metaestabilidad} Múltiples meta estados estacionarios, donde un pequeño cambio en términos pueden precipitar un cambio ilimitado en el sistema.
\item{\bf Irregularidad}  Distribución no gaussiana de los resultados, donde salidas que estan muy lejos del promedio son muy frecuentes.
\end{itemize}

\item[\bf{Problema 2}] Describe 3 ejemplos de sistemas complejos, explicando claramente por
qué sería útil considerarlos como complejos. Los ejemplos deben ser ejemplos de la vida diaria, no los clásicos que hay en toda la bibliografía.
\item[\bf{Respuesta}]

\begin{enumerate}
\item {\bf Comportamiento del mercado.}
El sistema económico es un sistema altamente correlacionado, donde los precios de los bienes están influenciados por bienes alternativos, sustitutos, por las tendencias, modas, etc.
Esto crea un sistema rico en interacciones y fuertemente codependiente.

\item {\bf Influencia social.}
Se conoce que hay ``lideres de opinion'', ``seguidores'', etc. Por la literatura del Marketing, aunque no solo restringido a los productos, sino a todo tipo de ideas, politicas, sociales, etc. Es un fenomeno muy complejo, y su estudio podría dar luz en temas de gran interes social y político.

\end{enumerate}
\end{enumerate}
\bibliographystyle{IEEEtran}
\bibliography{IEEEabrv,main}
%\bibliography{TareaExamen}
%\bibliographystyle{plain}
\end{document}

