\documentclass{article}
\usepackage{listings}
\usepackage[utf8]{inputenc}
\usepackage{ amssymb }
\usepackage{amsfonts}
\usepackage{graphicx}

\title{Sistemas Distribuidos y Verificación \\ Tarea 5}
\author{Fabián Romero Jiménez}
\date{}
\begin{document}
\maketitle
\begin{enumerate}

\item[{\bf Problema 1}] Recuerden la platica de Borzoo sobre program repair. Encuentra un algoritmo polinomial que repare la secuencia de estados, en el caso de una sola condición de progreso $\box(P \models \Diamond Q)$, en donde hay un ciclo y solo puedes quitar transiciones. Demuestra que es correcto y que es polinomial.

\item[\bf{Respuesta}] Suponiendo que la revisión de la condición de progreso puede revisarse en tiempo constante.\\
Sea $\mathcal{G=\{V,E\}}$ la gráfica de transiciones.

Para podar la gráfica se tendría el siguiente algoritmo:\\
\begin{enumerate}
\item Marcar todos los vértices excepto el inicial como no visitados\\
  complejidad $O(|\mathcal{V}|)$
\item Eliminar todos los vértices terminales donde no se cumple $\mathcal{Q}$\\
  complejidad $O(\mathcal{|V|+|E|})$
\item para cada vertice $v\in V . Q(v)=true$ que sea de estado terminal y marcado como no visitado: \\
  complejidad $O(|\mathcal{V}|* loop)$
\begin{enumerate}
\item Se busca un camino $P_v$ de longitud mínima desde el estado inicial $v$(podemos usar Dijkstra)
  complejidad $O(\mathcal{|E|+|V|}log|\mathcal{V}|)$
\item En la gráfica inducida $\mathcal{G}_P$ resultante al quitar de $\mathcal{G}$ las aristas de $P_v$ se busca otro camino de longitud mínima, si se encuentra, la última arista donde se intersecta con $P$ se elimina.\\
  complejidad $O(\mathcal{|V| * (|E|+|V|}log|\mathcal{V}|))$
\item Repetir el paso anterior hasta que no haya otro camino, y se marca $v$ como visitado
\item Eliminar todos los vértices terminales donde no se cumple $\mathcal{Q}$\\
  complejidad $O(\mathcal{|V|+|E|})$
\end{itemize}

en pseudocódigo, asumiendo que previamente esta implementada la función  $q$ que verifica la propiedad, el algoritmo de dijkstra y una funcion  last que dados dos caminos encuentra la ultima arista en el primero que incide con el segundo.
\newpage
\lstset{
  numbers=left,
  firstnumber=1,
  numberfirstline=true
}
\begin{lstlisting}[frame=single]
v.visitado=false for v in V where v != vo
V=V.filter( q(x) )
E=E.filter( (e)-> e.i in V and e.f in V )
while (v = V.find( q(v) and  
                   v.terminal and 
                   not v.visitado) != null):
    p=dijskstra(v0,v)
    E'=E.filter( (e)-> e not in p)
    while(q=dijskstra(v0,v) != null):
      E'=E'.filter( (e)-> e not in q)
      E = E - last(q,p)
    v.visitado=true
    V=V.filter( q(x) )
    E=E.filter( (e)-> e.i in V and e.f in V )
\end{lstlisting}

La complejidad del código se demustra basado en dos invarientes de los ciclos while.
el while interno (linea 9) cada vez que se ejecuta, el conjunto $\mathcal{E}$ disminuye en uno, por lo tanto no puede ejecutarse mas que $O(|E|)$.
el while externo (linea 4) cada vez que se ejecuta, en el conjunto $\mathcal{E}$ un elemento se marca como visitado y nunca mas se marca como no visitado de nuevo, por lo que no puede ejecutarse mas de $O(|V|)$ veces y como se sabe que dijsktra es correcto, sabemos que termina en un tiempo finito de hecho acotado por $O(|E|*|V|*O(dijkstra))$ y por lo tanto es tiempo polinomial.\\
Para ver que el resultado es correcto, solo falta notar que cada ejecución del ciclo while externo, se elimina una arista de un ciclo que lleva de $v_0$ a $v$, por lo que al final de la ejecución, queda un arbol, cuya raiz es el nodo inicial y todas las hojas estan en estados terminales donde $\mathcal{Q}$ se cumple, que es exactamente el objetivo del algoritmo $\blacksquare$






\item[\bf{Problema 2}] Explicar por qué el problema de encontrar un ciclo que pase por 2 vértices dados es NP-Completo. (Solo dar un esbozo de la prueba).\\

Sabemos que el problema de encontrar un ciclo hamiltoneano en una gráfica es NP completo.
Y reducimos el problema de encontrar un ciclo que pase por 2 vertices dados al problema del ciclo hamiltonaneo de la siguiente manera.\\

Supongamos que tenemos un algoritmo $c2v$ que encuentra un ciclo que pasa por 2 vertices dados. Dada una gráfica $G$ queremos encontrar un ciclo hamiltoneano.\\
tomamos $c2v$ y obtenemos un ciclo en $G$ (o una prueba de que no hay ningun ciclo).\\
Si encontramos el ciclo $C_1$, considere la gráfica $G_{C1}$ que es la gráfica inducida en $G$ al eliminar $C$ y todos los vértices por los que $C$ para, excepto por dos consecutivos cualesqiera en $C$ $v_{ic_1}$ y $v_{fc_1})$, y buscamos un ciclo que una $=(v_{ic_1},v_{fc_1})$,  si se encuentra, quiere decir que hay al menos 2 caminos entre $=(v_{ic_1},v_{fc_1})$ que no incluyen a $e_1$, por lo que podemos tomar $C_1$ y uno de esos caminos (uno tiene que ser diferente a $C_1$) tendríamos un ciclo más grande que $C_1$ en $G$ que no repite vértices, repetimos este proceso hasta que tengamos un ciclo que pase por todos los vértices de $G$.
 Este proceso no es más que aplicar a lo mas $|V|$ veces $c2v$, es decir podemos reducir el problema del ciclo hamiltoneano a $c2v$, por lo que este problema es NP completo.



\item[\bf{Problema 3}] Recordemos ahora el modelo de memoria compartida wait-free y asíncrono para n procesos. El algoritmo que ejecutan los procesos en primera instancia fue:

\begin{lstlisting}[frame=single]
Alg(id):
    r:= −1
    view := id
    loop:
        r := r+1
        mem[r] := write(view)
        X := scan(mem[r])
        view := conjunto de id s en X
    until |view|= n - r
output view
\end{lstlisting}
Después vimos una modificación al algoritmo, en el cual un proceso ``no olvida'' si en una iteración anterior vio a algún otro proceso. Cambiamos el código de la siguiente manera:
\begin{lstlisting}[frame=single]
Alg(id):
    r := − 1
    view := id
    loop:
        r := r+1
        mem[r] := write(view)
        X := scan(mem[r])
        view := view U conjunto de id’s en X
    until |view| >= n - r
output view
\end{lstlisting}.\\

Demuestra que los dos algoritmos son correctos y que cumplen la propiedad de que las vistas de los procesos están contenidas de acuerdo al orden en que los procesos terminan su iteración. (i.e. el proceso j terminó después que el proceso k, entonces la vista del proceso j está contenida en la vista del proceso k).

\item[\bf{Respuesta}]
Para demostrar la contención de las vistas en la ronda $j$ considere dos procesos $u$ y $v$, supongamos sin perdida de generalidad que $u$ corrió el ciclo por $j-$ésima vez, antes que $v$, por lo tanto, al ejecutar la linea 6, $u$ escribió y en la linea 8, leyo su propio estado y algunos otros, así, cuando $v$ ejecutó el ciclo $j-$ésima vez, leyo todo lo que leyo $u$ puesto que ningun proceso pudo cambiar lo que ya habia escrito en esa ronda, leyo tambien lo que escribio $u$ y por lo tanto $view_u \subseteq view_v$, esto es válido para ambos algoritmos.\\


Para demostrar que {\bf el primer algoritmo termina}, se toma como invariante la siguiente. Despues de $k$ rondas, $k$ procesos han terminado. Por inducción sobre $k$, el caso base es 0 y dice que despues de $0$ rondas $0$ procesos han terminado, lo cual es cierto, pues ni siquiera se ha comenzado.
Supongamos que hasta la ronda $k-1$ han salido $k$ procesos. (H de I)\\
Para la ronda $k$, han llegado $n-k+1$ procesos por hipotesis de inducción, ahora observe el último proceso en escribir en la ronda $k$, para el $|view|=n-k$ pues ya todos los demás estan en $view$ y  sólo para este, pues los demás vieron menos elementos, así entonces en la ronda $k$ sale un proceso más y por hipótesis de induccíon quedan exactamente $n-k$ procesos $\blacksquare$  

Para demostrar que {\bf el segundo algoritmo termina}, La prueba es muy similar al anterior, se toma como invariante la siguiente. Despues de $k$ rondas, $k$ procesos han terminado. Por inducción sobre $k$, el caso base es 0 y dice que despues de $0$ rondas $0$ procesos han terminado, lo cual es cierto, pues ni siquiera se ha comenzado.
Supongamos que hasta la ronda $k-1$ han salido $k$ procesos. (H de I)\\
Para la ronda $k$, han llegado $n-k+1$ procesos por hipotesis de inducción, ahora observe el último proceso en escribir en la ronda $k$, para el $|view| >=n-k$ pues ya todos los demás estan en $view$ y el se ve a si mismo, y sólo para este proceso la desigualdad es válida, pues los demás vieron menos elementos, así entonces en la ronda $k$ sale un proceso más y por hipótesis de induccíon quedan exactamente $n-k$ procesos $\blacksquare$  



\end{enumerate}
\end{document}
