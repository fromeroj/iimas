\documentclass{article}
\usepackage{listings}
\usepackage[utf8]{inputenc}
\usepackage{ amssymb }
\usepackage{amsfonts}
\usepackage{graphicx}

\title{Sistemas Distribuidos y Verificación \\ Tarea 3}
\author{Fabián Romero Jiménez}
\date{}
\begin{document}
\maketitle
\begin{enumerate}

\item[\bf{Problema 1}]  Recuerden el modelo visto en clase con el que resolvimos la tarea 
$\epsilon$-agreement para dos procesos. 
Alice y Bob proponen un valor y quieren quedar de acuerdo en un valor que no diste más $\epsilon$ de lo que el otro decidió. Después vimos que necesitaban comunicarse y que para $\epsilon$ más pequeñas se necesitaban más y más rondas de comunicación.
 Recuerdenque usamos el modelo de memoria compartida de lectura/escritura por capas full information (por cada escritura leiamos un nuevo arreglo y escribimos todo lo que sabemos cada vez).

Ahora proponemos otros dos modelos de comunicación:
El modelo chismoso y el modelo discusión civil. El modelo chismoso dice que si de Alice
y Bob alguno de los 2 no escucho al otro entonces tienen otra ronda de comunicación.
 Si los dos se escuchan entonces ahi temina la comunicación.

Podriamos verlo como que se lanzan insultos e indirectas pero no de frente.

El modelo discusión civil dice que mientras Alice y Bob esten escuchando mutuamente la conversación sigue. En el momento que uno ya no escucha al otro ahi se termina.

Sea $M \ge 0$ el número de rondas máximo que se pueden comunicar Alice
y Bob:

\begin{enumerate}
\item Para los 2 modelos dados describe el complejo del protocolo $M$. Tienes que dar la tripleta $( I, P_m , \Xi_ m)$
\item  Para los 2 modelos describe cuál es la tarea  $\epsilon$-agreement óptima para cada $M \ge 0$. Nos referimos a optimo como que cada vist del protocolo va a un valor de decisión único.
En otras palabras, encontrar $\epsilon$ en función de M.
\end{enumerate}

\item[\bf{Respuesta}]

\item[\bf{Problema 2}] Con el modelo visto en clase define la función de decisión de equidad de género $\delta$, que no distingue entre Alice y Bob. Es decir Alice y Bob pueden tener ya sea el valor 0 o 1 de entrada. También da la $\delta$ óptima para este caso. Aqui debes de tener cuidado de como etiquetas los vértices para que siempre se cumpla la $\epsilon$.

\item[\bf{Problema 3}] Ahora regresemos al modelo de memoria compartida y modifiquémoslo. Ahora supongamos que tenemos solo una memoria. Es decir ya no tenemos capas y sobrescribimos nuestra parte del arreglo cada ronda. También regresamos al caso en que Alice propone 0 y Bob 1.
\begin{itemize}
\item Describe el modelo (haz la gráfica) y ve como se ve la gráfica de vistas después de M rondas.
\item ¿Cuál es la mejor $\epsilon$ que se puede resolver en la tarea del $\epsilon$ agreement en M rondas?
\end{itemize}


\end{enumerate}
\end{document}
