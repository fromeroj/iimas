\documentclass{tufte-handout}

%\geometry{showframe}% for debugging purposes -- displays the margins
\usepackage[utf8]{inputenc}
\usepackage{amsmath}
\usepackage{graphicx}
\usepackage{amssymb}
\setkeys{Gin}{width=\linewidth,totalheight=\textheight,keepaspectratio}
\graphicspath{{graphics/}}
\title{Completness and consistency of a finite total order logic}
\author[Fabian Romero]{Fabian Romero}
\usepackage{booktabs}
\usepackage{units}
\usepackage{fancyvrb}
\fvset{fontsize=\normalsize}
\usepackage{multicol}
\usepackage{lipsum}
\newcommand{\doccmd}[1]{\texttt{\textbackslash#1}}
\newcommand{\docopt}[1]{\ensuremath{\langle}\textrm{\textit{#1}}\ensuremath{\rangle}}
\newcommand{\docarg}[1]{\textrm{\textit{#1}}}
\newenvironment{docspec}{\begin{quote}\noindent}{\end{quote}}
\newcommand{\docenv}[1]{\textsf{#1}}
\newcommand{\docpkg}[1]{\texttt{#1}}
\newcommand{\doccls}[1]{\texttt{#1}}
\newcommand{\docclsopt}[1]{\texttt{#1}}
\begin{document}

\maketitle

\begin{abstract}
\noindent
Consider two modalities ($\square$, $\blacksquare$) defined as $\square= $ $\mathcal{K}$ $ + $ L\"ob $+ .3 $\\ $\blacksquare = inverse(\square) + $ L\"ob, $\mathcal{L}$ the clasic propositional bimodal logic with those modalities. In this work it is established that $\mathcal{L}$ is consistent, complete and that is determined by all rooted frames $\mathfrak{F}$  that are finite total orders.
\end{abstract}

\section{Introduction}\label{sec:Forewor}
\subsection{Modal logic}\label{sec:lts}

Historically known as the study of ``possibility'' and ``necessity'' modal logic can be traced back to the classic Aristotelic works about 24 centuries ago. The usage of transition structures to study and model them is far more recent. Introduced by Samuel Kripke on his article ``Semantical Considerations on Modal Logic''\cite{Kri63}, conceptually based on a mental experiment by Leibniz, but extending this idea to make it a usefull practical tool for exploring and defining modalities.\\

The work of Kripke among many others in the 60s and 70s reshaped and extended dramatically the study of modal logic, the introduction of transition structures and many other methods and techniques for its study and the characterization lead to a plethora of results and the study of many modal logics and the frame structures they realize.\\

Soon, mathematicians and philosophers alike, started using those tools to reason about different modalities, enriching modalities such as necessity, obligation, knowledge, belief, temporality, provability, etc.\\




Some authors of modern books on modal logic such as Blackburn and ``Sally pookorn'' (it is a pseydonym).

The work of Kripke among many others in the 60s and 70s reshaped and extended dramatically the study of modal logic, the introduction of transition structures and many other methods and techniques for its study and the characterization lead to a plethora of results and the study of many modal logics and the fra me structures they realize.\\

Soon, mathematicians and philosophers alike, started using those tools to reason about different modalities, enriching modalities such as necessity, knowledge, belief, obligation, temporality, provability, etc.\\

However, labelled transition structures occur not only on logic, but they often apear in mathematics (graph theory, topology, Markov chains, categories,  partial orders, equivalence relations, etc.) and in computer science ( automata, process algebras, BPNM,  etc. ).\\

The preponderancy and importance of this structures, are central on the modern study of modal  logic. Or, as ``Sally Popcorn'' boldly stated:

The objective of modal logic us not an analysis of modal languages; is it is not the construction of different proff styles and rules of inference, etc. The main objective of modal logic is, no more and no less, the study of labelled transition structures. The modal language and all its attachments is there simply to help out.\\



In logic, transition structures are often asociated with modal logics and in the case of mono modal logics asociated with

There are few results on characterization of logics that finite frames



\section{Support}\label{sec:support}

The website for the Tufte-\LaTeX\ packages is located at
\url{http://code.google.com/p/tufte-latex/}.  On our website, you'll find
links to our \smallcaps{svn} repository, mailing lists, bug tracker, and documentation.

\bibliography{sample-handout}
\bibliographystyle{plainnat}



\end{document}
