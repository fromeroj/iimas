\newcommand{\ds}{\textbf{\textit{DS}}\xspace}
\newcommand{\sh}{\textbf{\textit{WFSM}}\xspace}
\documentclass[hyperref={pdfpagelabels=false}]{beamer}
\usepackage[utf8]{inputenc}
\usepackage{listings}
% By  using hyperref={pdfpagelabels=false} you get rid off:
% Package hyperref Warning: Option `pdfpagelabels' is turned off
% (hyperref)                because \thepage is undefined. 
% Hyperref stopped early 
%

\usepackage{lmodern}
% Using lmondern and you get rid off this:
% LaTeX Font Warning: Font shape `OT1/cmss/m/n' in size <4> not available
% (Font)              size <5> substituted on input line 22.
% LaTeX Font Warning: Size substitutions with differences
% (Font)              up to 1.0pt have occurred.
%

% If \titel{$B!D(B} \author{$B!D(B} come after \begin{document} 
% you get the following warnig:
% Package hyperref Warning: Option `pdfauthor' has already been used,
% (hyperref) ... 
% So it is here before \begin{document}

\usetheme{AnnArbor}
\usecolortheme{beaver}
\title{ Characterization of topological models for distributed computing with logical systems. }   
\author{Fabian Romero} 
\date{}
% additional usepackage{beamerthemeshadow} is used
%\usepackage{beamerthemeshadow}

\begin{document}
\lstdefinelanguage{json}{
    basicstyle=\normalfont\ttfamily,
    numbers=left,
    framexleftmargin=0pt,
    xleftmargin=2em,
    numberstyle=\scriptsize,
    stepnumber=1,
    numbersep=8pt,
    showstringspaces=false,
    breaklines=true,
    morecomment=[s]{/*}{*/},
    morecomment=[l]--,
    frame=lines,
%    backgroundcolor=\color{background},
    literate=
     *{0}{{{\color{numb}0}}}{1}
      {1}{{{\color{numb}1}}}{1}
      {2}{{{\color{numb}2}}}{1}
      {3}{{{\color{numb}3}}}{1}
      {4}{{{\color{numb}4}}}{1}
      {5}{{{\color{numb}5}}}{1}
      {6}{{{\color{numb}6}}}{1}
      {7}{{{\color{numb}7}}}{1}
      {8}{{{\color{numb}8}}}{1}
      {9}{{{\color{numb}9}}}{1}
      {:}{{{\color{punct}{:}}}}{1}
      {,}{{{\color{punct}{,}}}}{1}
      {\{}{{{\color{delim}{\{}}}}{1}
      {\}}{{{\color{delim}{\}}}}}{1}
      {[}{{{\color{delim}{[}}}}{1}
      {]}{{{\color{delim}{]}}}}{1},
}

\begin{frame}
\titlepage
\end{frame} 
 
\section{Synopsis} 
\subsection{Synopsis} 
\begin{frame}
\frametitle{Synopsis}
The objective of this work is to study the topological model for distributed computing from a formal logic perspective by mapping those topological models to Kripke frames and then, by applying model checking and other techniques. Specifically targeting consensus problem as a subject of study.
\end{frame}

\section{Introduction} 
\subsection{Introduction} 
\begin{frame}
\frametitle{Introduction}
  \begin{itemize}[<+->]
  \item Operational definition of Distributed systems.
  \item Why is this complex?.
  \item Herlihy's combinatiorial definition.
  \item logic and distributed systems.
  \item this work.
  \end{itemize}
\end{frame}

\section{Background} 
\subsection{Distributed computing, wait free shared memory model} 
\begin{frame}
\frametitle{Distributed computing, wait free shared memory model}
  \begin{itemize}[<+->]
  \item There is a partial order on the set of all system events (Lamport).
  \item {\bf Order-extension principle} states that a total order can be found as a linear extension for any partial order.
  \item Normal form wait free protocol
  \item Partial order induced is temporal order
  \end{itemize}


\end{frame}



\end{document}
