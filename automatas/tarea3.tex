\documentclass{article}
\usepackage{tikz}
\usepackage[utf8]{inputenc}
\usepackage{ amssymb }
\usepackage{amsfonts}
\usetikzlibrary{automata,positioning,arrows}

\title{Autómatas y lenguajes formales. Tarea 3}
\author{Fabián Romero Jiménez}
\begin{document}
\maketitle
\begin{enumerate}
\item[\bf{Problema 1}] Diseña una máquina de Turing que reconozca el lenguaje $\{\alpha\alpha\alpha | \alpha \in \{a, b\}^\star \}$

\begin{tabular}{|l||l|l|l|l|l|l|l|l|l|l|}
  \hline
   & $\vdash$ &$a$& $b$&$a_1$ & $b_1$ & $a_2$ & $b_2$  & $\dashv$ \\
  \hline  
  \hline  
  $s$      & $s,\vdash,\rightarrow$ &  $s_1,a_1,\rightarrow$ &  $s_1,b_1,\rightarrow$ &$s,a_1,\rightarrow$ &$s,b_1,\rightarrow$ &  &  & Acepta \\  
  \hline  
  $s_1$   &  &  $s_1,a,\rightarrow$ &  $s_1,b,\rightarrow$ & $s_1,a_1,\rightarrow$ &$s_1,b_1,\rightarrow$ &  $s_2,a_2,\leftarrow$ &  $s_2,b_2,\leftarrow$  & $s_2,\dashv,\leftarrow$  \\
  \hline  
  $s_2$    &  &  $s_3,a_2,\leftarrow$ &  $s_3,b_2,\leftarrow$ &  &  & & &\\  
  \hline  
  $s_3$    &  &  $s_4,a_2,\leftarrow$ &  $s_4,b_2,\leftarrow$ &  &  & & &\\  
  \hline  
  $s_4$    &  &  $s_5,a,\leftarrow$ &  $s_5,b,\leftarrow$ & $v_1,a_1,\leftarrow$  & $v_1,b_1,\leftarrow$  & & &\\  
  \hline  
  $s_5$    &  &  $s_5,a,\leftarrow$ &  $s_5,b,\leftarrow$ & $s,a_1,\rightarrow$  & $s,b_1,\rightarrow$  & & &\\
  \hline
  $v_1$    & $v_2,\_,\rightarrow$ & & & $v_1,a_1,\leftarrow$  & $v_1,b_1,\leftarrow$  & & &\\  
  \hline
  $v_2$    & &   &   & $v_{a1},\vdash,\rightarrow$ &  $v_{b1},\vdash,\rightarrow$ & & &  \\  
  \hline  
  $v_{a}$      &  &  & & $v_{a},a_1,\rightarrow$ &  $v_{a},b_1,\rightarrow$ & $v_1,a_1,\leftarrow$ &  & Acepta \\  
  \hline  
  $v_{b}$      &  &  & & $v_{a1},a_1,\rightarrow$ &  $v_{a1},b_1,\rightarrow$ & &$v_1,b_1,\leftarrow$  & Acepta \\  
  \hline  
\end{tabular}

\item[\bf{Explicación}]
Se marca un carácter al inicio (le ponemos un subindice 1) y dos al final (con subindice 2) y se repite el proceso hasta que no haya carácteres sin marcar.
Si no acaba exactamente la máquina se detiene en no aceptación, si acaba exactamente la cadena es de longitud $3k, k \in \mathbb{N}$ y hay $k$ carácteres al principio marcados (1) y $2k$ carácteres marcados al final (2). Los estados que hacen la operación de marcar son los estados $s_i$.

Una vez marcados todos borramos el primer carácter en la cadena y vamos a un estado donde buscamos el primer carácter con subindice 2, si es el mismo carácter base que el borrado, cambiamos de subindice (2) a subindice (1) y regresamos al principio de la cadena y repetimos, así al final de $2k$ operaciones si se eliminaron todas los subindices 2 y aceptamos la cadena. 


\item[\bf{Problema 2}] Diseña una máquina de Turing que acepte el conjunto ${a^{2^m}, m \in \mathbb{Z}^+}$.

\begin{tabular}{|l||l|l|l|l|l|l|l|l|l|l|}
  \hline
   & $\vdash$ &$a$& $a_1$ & $a_2$ & $\dashv$ \\
  \hline  
  \hline  
  $s$      & $s,\vdash,\rightarrow$ &  $s_1,a,\rightarrow$ &  &  &  \\  
  \hline
  $s_1$    &  &  $s_2,a,\leftarrow$ &  &  Acepta  & Acepta  \\
  \hline  
  $s_2$    &  &  $s_3,a_1,\rightarrow$ &  &    &   \\
  \hline  
  $s_3$   &  &  $s_3,a,\rightarrow$ &  &  $s_4,a_2,\leftarrow$  & $s_4,\dashv,\leftarrow$  \\
   \hline  
  $s_4$   &  &  $s_5,a_2,\leftarrow$ &  &  &  \\
  \hline  
  $s_5$  &   &  $s_6,a,\leftarrow$  & $s_7,a,\leftarrow$  &   &  \\
  \hline  
  $s_6$  &   &  $s_6,a,\leftarrow$  & $s_2,a_1,\rightarrow$  &   &  \\
  \hline  
  $s_7$  & $s,\vdash,\rightarrow$  &   & $s_7,a,\leftarrow$  &   &  \\
  \hline  
\end{tabular}

\item[\bf{Explicación}] Se verifica si la cadena no marcada es de longitud 1, en este caso se acepta (Estados $s,s_1$), luego se marca el primer carácter no marcado con subindice 1 y el ultimo carácter no marcado con subindice 2 y se repite el proceso hasta que no hay más carácteres por marcar, partiendo el conjunto inicial de carácteres no marcados en carácteres marcados con la primera mitad marcada con subindice 1 y la segunda marcada con subindice 2, luego se borran las marcas a todos los elementos con subindice 1 y ser repite el procedimiento hasta que termine de verificar la cadena.

\item[\bf{Problema 3}] Demuestra que el conjunto $TOT = \{ M | M $ se detiene con todas las entradas$ \}$ no es recursivamente enumerable y tampoco lo es su complemento.

\item[\bf{Demostración}] Por diagonalización:\\

Supongamos que si, que $TOT = \{ M | M $ se detiene con todas las entradas$ \}$ es recursivamente enumerable, por lo que hay una máquina de Turing $M_{TOT}$ que puede generar todas las máquinas en $TOT$, creemos una matriz cuadrada con $M_i$ la $i$-ésima máquina total generada por $M_{TOT}$ poniendo en $(i,j)$ si la máquina $M_i$ acepta o rechaza $M_j$, esto es posible por que $M_i$ es total.

\begin{tabular}{|l||l|l|l|l|}
  \hline
          & $M_1$ & $M_2$ & $M_3$ & $...$ \\
   $M_1$  & $[0]$   & $1$   & $0$   & $...$ \\
   $M_2$  & $1$   & $[1]$   & $1$   & $...$ \\
   $M_3$  & $0$   & $1$   & $[1]$   & $...$ \\
   $...$  & $...$ & $...$ & $...$ & $[...]$ \\ 
  \hline  
\end{tabular}



 por lo tanto hay una maquina de turing $TM_{TOT}$ que acepta el lenguage $TOT$ y podemos crear una maquina $TM_{NO} = \{M | M \in TOT \wedge M$ no acepta $M \}$ simplemente poniendo la maquina que evalua a una maquina con ella misma como entrada después de pasarla por $TOT$, como $TOT$ regresa solo maquinas totales cuando evaluamos $M$ en $M$ termina eventualmente por lo que $TM_{NO}$ es entonces recursivente enumerable.
ahora $TM_{NO}$ se detiene con $TM_{NO}$?,  si 

 

\end{enumerate}


\end{document}  
