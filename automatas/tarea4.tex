\documentclass{article}
\usepackage{tikz}
\usepackage[utf8]{inputenc}
\usepackage{ amssymb }
\usepackage{amsfonts}
\newcommand\defeq{\stackrel{\mathclap{\normalfont\mbox{def}}}{=}}
\usetikzlibrary{automata,positioning,arrows}

\title{Autómatas y lenguajes formales. Tarea 4}
\author{Fabián Romero Jiménez}
\begin{document}
\maketitle
Considera las siguientes funciones:\\
Factorial: \\
$0! = 1$ \\
$(n + 1)! = n!  (n + 1)$ \\
Exponenciación \\
$n^0=1$ \\
$n^{m+1}=n^m \times n$ \\

\begin{enumerate}
\item[\bf{Problema 1}] Formula estas funciones en el formato de las  
funciones recursivas $\mu$ 

\item[\bf{Factorial}]
$Factorial(0)  \stackrel{def}{\equiv} 1$\\
$Factorial(n+1) \stackrel{def}{\equiv} mult(n+1,Factorial(n))$

\item[\bf{Exp}]
$Exp(n,0)  \stackrel{def}{\equiv} 1$\\
$Exp(n,m+1) \stackrel{def}{\equiv} mult(n,Exp(n,m))$

\item[\bf{Problema 2}] Transforma tus definiciones anteriores de acuerdo con las convenciones vistas para el cálculo $\lambda$.

\item[\bf{Factorial}]

$Factorial \stackrel{def}{\equiv}  Y \lambda f . \lambda n.(Cero \: n)(\bar 1)(MULT(n)(f(P n))) $\\

\item[\bf{Exp}]$Exp \stackrel{def}{\equiv}  Y \lambda f . \lambda n,m.(Cero \: m)(\bar 1) (MULT(n)(f\:n\:(P m) )) $\\

\item[\bf{Problema 3}] Defínelas por medio de programas URM.

\item[\bf{Problema 4}]Define la semántica del comando
for $i := 1$ to n do P
y utilízalo para definir la recursión primitiva de una manera más directa

\item[\bf{Problema 4}] Define la minimalización no acotada por medio de programas $while$.


1:J(2,3,0)
2:


1:J(6,8,9)
2:J(5,7,6)
3:S(4)
4:S(7)
5:J(1,1,1)
6:Z(7) --- salto
7:S(8)
8:J(1,1,1)
9:T(4,1)

------------

1:J(6,8,0)
2:J(5,7,6)
3:S(4)
4:S(7)
5:J(1,1,1)
6:Z(7) --- salto
7:S(8)
8:J(1,1,1)
9:T(4,1)


---------
      
1:J(3,5,0)
2:J(2,4,6)
3:S(1)
4:S(4)
5:J(1,1,1)
6:Z(4)
7:S(5)
8:J(1,1,1)


\end{enumerate}
\end{document}
