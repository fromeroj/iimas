\documentclass{article}
\usepackage{listings}
\usepackage[utf8]{inputenc}
\usepackage{ amssymb }
\usepackage{amsfonts}

\title{Autómatas y lenguajes formales. Tarea 4}
\author{Fabián Romero Jiménez \\ Israel Sandoval Grajeda}
\begin{document}
\maketitle
Considera las siguientes funciones:\\
Factorial: \\
$0! = 1$ \\
$(n + 1)! = n!  (n + 1)$ \\
Exponenciación \\
$n^0=1$ \\
$n^{m+1}=n^m \times n$ \\

\begin{enumerate}
\item[\bf{Problema 1}] Formula estas funciones en el formato de las  
funciones recursivas $\mu$ 

\item[\bf{Factorial}]
$Factorial(0)  \stackrel{def}{\equiv} 1$\\
$Factorial(n+1) \stackrel{def}{\equiv} mult(n+1,Factorial(n))$

\item[\bf{Exp}]
$Exp(n,0)  \stackrel{def}{\equiv} 1$\\
$Exp(n,m+1) \stackrel{def}{\equiv} mult(n,Exp(n,m))$

\item[\bf{Problema 2}] Transforma tus definiciones anteriores de acuerdo con las convenciones vistas para el cálculo $\lambda$.\\
Sean: \\
$H \stackrel{def}{\equiv}  \lambda x_1.x_1$ \\
$MULT \stackrel{def}{\equiv} Y \lambda f . \lambda x,y . (Cero\:x)(H\bar0)(SUMA(f(P x)y)y)$\\$\bar1 = S(\bar0)$\\

\item[\bf{Factorial}]

$Factorial \stackrel{def}{\equiv}  Y \lambda f . \lambda n.(Cero \: n)(H\bar 1)(MULT(n)(f(P n))) $\\

\item[\bf{Exp}]$Exp \stackrel{def}{\equiv}  Y \lambda f . \lambda n,m.(Cero \: m)(H\bar 1) (MULT(n)(f\:n\:(P m) )) $\\

\item[\bf{Problema 3}] Defínelas por medio de programas URM.

Este programa usa 6 registros, los valores iniciales son:
$1,n,0,0,0,0$ donde $n$ es el numero del que se quiere el factorial y el resultado
será guardado en el registro $1$ \\

\item[\bf{Factorial}]
\begin{lstlisting}[frame=single] 
1:S(3)     -- inc contador fact
2:J(2,3,0) -- Detencion al finalizar
3:T(1,4)   -- Res -> multiplicando
4:Z(6)     -- Reset contador mult
5:Z(5)     -- Reset contador suma
6:J(4,6,1) -- mult concluida cont fact 
7:J(3,5,11)-- suma concluida cont mult
8:S(1)     -- inc result
9:S(5)     -- inc contador suma
10:J(1,1,6)-- iterar suma
11:S(6)    -- inc contador mult
12:J(1,1,5)-- iterar mult
\end{lstlisting}


\item[\bf{Exp}]

Este programa usa 7 registros, los valores iniciales son:
$1,n,m,0,0,0,0$ donde $n$ es el numero a elevar a la potencia $m$ y el resultado
será guardado en el registro $1$ \\

\begin{lstlisting}[frame=single] 
1:J(3,4,0) -- Detencion al finalizar
2:S(4)     -- inc contador pow
3:T(1,5)   -- Res -> multiplicando
4:Z(7)     -- Reset contador mult
5:Z(6)     -- Reset contador suma
6:J(5,7,1) -- mult concluida cont pow
7:J(2,6,11)-- suma concluida cont mult
8:S(1)     -- inc result
9:S(6)     -- inc contador suma
10:J(1,1,6)-- iterar suma
11:S(7)    -- inc contador mult
12:J(1,1,5)-- iterar mult
\end{lstlisting}

\newpage
\item[\bf{Problema 4}]Define la semántica del comando
for $i := 0$ to n do P
y utilízalo para definir la recursión primitiva de una manera más directa

Se define:\\
\begin{lstlisting}[frame=single] 
for i := 0 to n do P
\end{lstlisting}

como:\\
\begin{lstlisting}[frame=single] 
i:=0;              
while i < n  do      
    P
    i=i+1;
\end{lstlisting}

Así la definición de recursión primitiva con: \\
$[Ph]\sigma(X_0) = h(\sigma(X_1))$\\
$[Pg]\sigma(X_0) = g( \sigma(Z),\sigma(X_1),\sigma(X_2))$\\
queda definido por:\\

\begin{lstlisting}[frame=single] 
W:=Z;
Y1=X1;
Ph;
for I:=0 to W do
  Z:=I;
  X1=Y1;
  X2=X0;
  Pg
\end{lstlisting}

\newpage
\item[\bf{Problema 5}] Define la minimalización no acotada por medio de programas $while$.

La minimalización no acotada se define como:


Sea $ g: \mathbb{N}^{n+1} \rightarrow  \mathbb{N} $ una función computable.
Entonces la función\\
$f: \mathbb{N}^{n+1} \rightarrow \mathbb{N}$ definida como:\\
$f(\bar x)= min\{m \in \mathbb{N} |g(x,m)=0, \forall i \le m . g(x,n) \}$
o indefinida si el conjunto es vacío. En clase vimos la existencia y consistencia de tal definición.

Y suponiendo la existencia del programa $[Pg]$\\
$[Pg]\sigma(X_o,X_1) \rightarrow g(\sigma(X_0),\sigma(X_1))$

Lo cual podemos expresar con el siguiente programa while:\\

\begin{lstlisting}[frame=single] 
M:=0;                %El minimo valor
Z:=False;           
while Z=False do     %Si no todos son ceros
    Z:=True;         %En el conjunto vacio es cierto
    I:=0;            %El indice desde 0 hasta M
    M:=M+1;         
    X0=M;            %El valor de salida es M 
    while (M>I) do   %Buscando desde 0 a M
      Y0=C0;         %Primer parametro de Pg
      Y1=C1;         %Segundo parametro de Pg
      Pg;            %Evaluacion de la funcion       
      Z=Z and (X0=0) %si el resultado es 0 hasta I
      I=I+1          %Incrementamos I
\end{lstlisting}
Observese que aunque la función $[Pg]$ es total, la minimización no necesariamente termina, pues requiere de hacer la búsqueda del mínimo sin cota superior.
\end{enumerate}
\end{document}
