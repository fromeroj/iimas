\documentclass{article}
\usepackage{tikz}
\usepackage[utf8]{inputenc}
\usepackage{ amssymb }
\usepackage{amsfonts}
\newcommand\defeq{\stackrel{\mathclap{\normalfont\mbox{def}}}{=}}
\usetikzlibrary{automata,positioning,arrows}



\title{Autómatas y lenguajes formales. Tarea 4}
\author{Fabián Romero Jiménez}
\begin{document}
\maketitle
Considera las siguientes funciones:\\
Factorial: \\
$0! = 1$ \\
$(n + 1)! = n!  (n + 1)$ \\
Exponenciación \\
$n^0=1$ \\
$n^{m+1}=n^m \times n$ \\

\begin{enumerate}
\item[\bf{Problema 1}] Formula estas funciones en el formato de las  
funciones recursivas $\mu$ 

\item[\bf{Factorial}]
$Factorial(0)  \stackrel{def}{\equiv} 1$\\
$Factorial(y+1) \stackrel{def}{\equiv} mult(y+1,Factorial(y))$

\item[\bf{Exp}]
$Exp(x,0)  \stackrel{def}{\equiv} 1$\\
$Exp(x,y+1) \stackrel{def}{\equiv} mult(x,Exp(x,y))$

\item[\bf{Problema 2}] Transforma tus definiciones anteriores de acuerdo con las convenciones vistas para el cálculo $\lambda$.

\item[\bf{Factorial}]


$FACTORIAL \stackrel{def}{\equiv} Y\lambda f.$
\item[\bf{Problema 3}] Defínelas por medio de programas URM.

\item[\bf{Problema 4}]Define la semántica del comando
for $i := 1$ to n do P
y utilízalo para definir la recursión primitiva de una manera más directa

\item[\bf{Problema 4}] Define la minimalización no acotada por medio de programas $while$.

\end{enumerate}


\end{document}  
