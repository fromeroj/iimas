\documentclass[12pt]{article}
\usepackage[utf8]{inputenc}
\usepackage{listings}
\usepackage{color}
\title{ On finite, total order kripke frames }
\author{Fabián Romero}
\date{}
\begin{document}
\maketitle

\begin{abstract}
Your abstract goes here...
...
\end{abstract}

\begin{itemize}
  \item[1] Demuestra que todo torneo tiene un rey y un súbdito. Es decir, un vértice desde el cual se puede llegar a todos los demás vértices pasando a lo más por dos arcos, y uno al cual todos pueden llegar pasando a lo más por dos arcos

  \item[Respuesta:] Demostremos por inducción que cada torneo tiene un rey, el caso base es la gráfica de 1 vertíce, por vacuidad, el único vertice es rey.\\
    Asumamos que es cierto que cualquier torneo de $n$ vertices tiene un rey, demostremos ahora el caso del torneo de $n+1$ vértices.
    Sea $G$ un torneo de $n+1$ vértices y tomemos un vértice cualquiera $v$, entendamos $G_{-v}$ la grafíca $G$ eliminando $v$ y todas las aristas incidentes.



\end{itemize}
\end{document}
