\documentclass{article}
\usepackage{listings}
\usepackage[utf8]{inputenc}
\usepackage{ amssymb }
\usepackage{amsfonts}

\title{Automatas Celulares \\ Tarea 1}
\author{Fabián Romero Jiménez}
\begin{document}
\maketitle

\begin{enumerate}

\item[\bf{Problema 1}] Da un ejemplo de un sistema continuo, uno discreto, uno orientado a eventos y uno combinado.\\

\item[{Continuo}] El clima, este sistema evoluciona continuamente en el tiempo (pues cambia de un momento a otro de forma continua) y tambien es continuo en el espacio, asi es un ejemplo que tanto en espacio y en tiempo varia continuamente una variable continua (la temperatura)\\

\item[{Discreto}] Un sistema discreto, podria ser el control de población en una región geogáfica. Pues no se toma para el la variable estocástica del momento de nacimiento, ni la movilidad de la persona, solamente se considera la variable discreta de fecha de nacimiento y lugar de residencia declarada.\\

\item[{Orientado a eventos}] Un sistema orientado a eventos podría ser la medición de sismos de intensidad mayor a 4 en la escala de Richter, puesto que el momento en que suceden no es predecible y no hay forma conocida de calcularlos ni una frecuencia o patrón conocido.\\

\item[{Mixto}] Un sistema mixto, podría ser el descrito en el problema anterior pero sin limitarlo a una intensidad, es decir, el de mantener la actividad sismica de una región geográfica, aqui, podemos ver problemas Orientados a eventos como la ocurrencia de un movimiento de tal intensidad, o continuos, como el registro de la actividad sismica en todo momento.\\


\item[\bf{Problema 2}] Da un ejemplo de un sistema dinámico y uno estático.\\

\item[{Dinámico}]  Casi cualquier sistema en la naturaleza, es uno dinámico, pues cambian con el tiempo, ejemplo, el flujo de los rios.\\

\item[{Estático}] El sistema de deducción lógica, que representa un sistema teótico, es estático, pues el conjunto de elementos que presenta (deducciones, teoremas, etc.) son independientes del tiempo.\\

\item[\bf{Problema 3}] Da tres ejemplos de aplicación de simulación en la realidad. Debes especificar el problema claramente.\\

\item[{Sistemas SCADA}]  En muchos sistemas de control (SCADA) se tiene un número de sensores que identifican las condiciones del sistema, muchos de estos sistemas se ponen ademas simuladores para poder entender y predecir el comportamiento en diversas situaciones.\\

\item[{Aplicaciones Económicas}] Para identificar el valor de un bien en el fututo, se deben de plantear escenarios y correr simulaciones que nos permitan entender y calcular dicho valor con tal de poder establecer su valor en el tiempo.\\

\item[{Ingeniería Civil}] En este caso, también se plantean escenarios posibles y se simulan sobre una o un conjunto de estructuras para ver cual sería el efecto de dicho escenario sobre la estructura.\\

\end{enumerate}
\end{document}
