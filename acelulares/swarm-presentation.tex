\documentclass[hyperref={pdfpagelabels=false}]{beamer}
\usepackage[utf8]{inputenc}
\usepackage{mathalfa}
% By  using hyperref={pdfpagelabels=false} you get rid off:
% Package hyperref Warning: Option `pdfpagelabels' is turned off
% (hyperref)                because \thepage is undefined.
% Hyperref stopped early
%

\usepackage{lmodern}
% Using lmondern and you get rid off this:
% LaTeX Font Warning: Font shape `OT1/cmss/m/n' in size <4> not available
% (Font)              size <5> substituted on input line 22.
% LaTeX Font Warning: Size substitutions with differences
% (Font)              up to 1.0pt have occurred.
%

% If \titel{$B!D(B} \author{$B!D(B} come after \begin{document}
% you get the following warnig:
% Package hyperref Warning: Option `pdfauthor' has already been used,
% (hyperref) ...
% So it is here before \begin{document}

\usetheme{AnnArbor}
\usecolortheme{beaver}
\title{ Verifying emergent properties of swarms }
\author{Panagiotis Kouvaros and Alessio Lomuscio }
\date{}
% additional usepackage{beamerthemeshadow} is used

%\usepackage{beamerthemeshadow}

\begin{document}

\begin{frame}
\titlepage
\end{frame}

\section{Introduction}

\subsection{Introduction}
\begin{frame}
\frametitle{Swarms}
\begin{itemize}
\item Many behaviourally identical agents
\item some may fail
\item focus on {\it emergent behaviour}.\\
  Trait of the swarm as a whole possibly not predictablefrom the behaviour of the individual agents (herd behaviour, flocking, etc.)
\item establish a lower bound on the number of agents
\item sound and complete!
\end{itemize}
\end{frame}

\begin{frame}
\frametitle{Emergent behaviour}
\begin{itemize}

\item many behaviourally identical agents
\item some may fail
\item we focus on emergent behaviour
\end{itemize}
\end{frame}


\begin{frame}
\frametitle{Assumptions}
\begin{itemize}
\item discrete arena in two spacial dimensions (+ time)
\item limited signal reach
\item can be extended to thee spacial dimentions
\item arena with periodic boundaries
\end{itemize}
\end{frame}


\begin{frame}
  \frametitle{ Model }
    $$\mathcal{R}(\alpha , \beta ) = (L, I, Act, P, t)$$
\begin{itemize}
\item[$\alpha$] arena size
\item[$\beta$] signal reach
\item[L] $\{1,..,\alpha\} \times \{1,..,\alpha\} \times S$
\item[]  neighbourhood - Chebyshev distance
\end{itemize}
\end{frame}


\begin{frame}
  \frametitle{  Chebyshev distance }
Chebyshev (discrete) distance with periodic boundaries.
\includegraphics[width=200px]{chev.png}
\end{frame}

\begin{frame}
  \frametitle{ Model }
Protocol (Action on a local state):
$$P : L \rightarrow \mathcal{P}(Act)$$
Transition function:
$$ t : L \times L_E \times Act \times \mathcal{P}(Act) \times Act_E \rightarrow L $$\\
Takes agent's state, environment's state, agent, neighbours and environment actions to determine next agent state.
\end{frame}

\section{ Swarm Systems}
\begin{frame}
  \frametitle{ Swarm System }
  Environment:
  $$ \mathcal{E} = (L_E, I_E, Act_E. P_E, t_E)$$
  No empty set $I_E$ initial states, states $L_E$ with $I_E  \subseteq L_E$
, and a nonempty set of actions ActE. The environment
E also includes a protocol $PE : LE \rightarrow P(ActE)$,
and a transition function $$ t_E : L_E \times Act_E \times P(Act) \rightarrow L_E$$
Swarm System:
$$S = \langle \mathcal{R,E,V} \rangle$$
template agent, template environment and a labeling function $\mathcal{V}: L \times L_E  \rightarrow \mathcal{P}(AP)$ to a set $AP$ of atomic propositions.
\end{frame}

\begin{frame}
  \frametitle{ Swarm System }
  for $n>0$ the concrete swarm $S(n)$ will encode precisely n agents interacting with the environment.\\
Let $\{1,...,n\}$ the set of the concrete agents in
$S(n)$. For any $i \in \{1,...,n\}$, the i-th  agent $R_i =(L_i,I_i,Act_i,P_i,t_i)$ is\\
$L\i= L \times \{i\}$,$I=i=I \times  \{i\}$, $Act_i=Act$, $P_i$ described as $a \in P_i(l,i)$ iff $a \in P(l)$ and $t_i((l,i),l_E,a,A,a_E=(l',i)$ iff $t(l,a_E,A,a_E)=l'$\\
Basically, an indexation by the agent's number.\\
A global state\\ $$ g = (l_1,..., l_n, l_E)$$
is a snapshot of the environment and all the agents.\\
denote $\mathfrak{N}(g,i)$ the set of agents in the neighbourhood of i when $S(n)$ is at a global state g.
\end{frame}


\begin{frame}
  \frametitle{ Global transition relation }
  $ \mathcal{T}(n) \subseteq G(n) \times ACT(n) \times G(n)$ on a set $G(n)$ of concrete global states on $S(n)$ as $(g,\bar{a},g')$ in $\mathcal{T}(n)$ iff
  $\bar{a}.E \in P_E(g.E and t_E(g.E,bar{a}.E,A)) = g'.E$ with $A=\{\bar{a}.i | i \in \{1,...,n\}\}$

\end{frame}

\begin{frame}
  \frametitle{  }
\end{frame}

\begin{frame}
  \frametitle{  }
\end{frame}

\end{document}
