\documentclass[12pt]{article}
\usepackage[utf8]{inputenc}
\usepackage{listings}
\usepackage{ amssymb }
\usepackage{color}

\title{Estructura de datos y teoría de algoritmos\\
Tarea 1 }
\begin{document}
\lstset{language=python}
\maketitle

\section{Tarea 1}

\begin{itemize}
  \item[\bf{Pregunta 1}] Demuestra que todo torneo $T$ tiene un rey y un súbdito. Es decir, un vértice desde el cual se puede llegar a todos los demás vértices pasando a lo más por dos arcos, y uno al cual todos pueden llegar pasando a lo más por dos arcos

  \item[Demostración:]
    Sea $T=(V,E)$ un torneo y $ v \in V $ tal que $|out(v)| \ge |out(u)| \forall u\in V $ es decir la cardinalidad de la exvecindad de $v$ es máxima en $T$\\
    denotaremos $x \rightarrow  y, si (x,y)\in E$\\
    Afirmación:\\
    $v$ es un rey en $T$\\
    Es decir $$\forall u\ne v \in V ,  (v\rightarrow u) \vee (\exists u_v \in V  |  (v \rightarrow u_v) \wedge (u_v \rightarrow u))$$
    Demostración por reducción al absurdo:\\
    Supongamos que no es cierto, es decir:
    $$\exists u\ne v \in V ,  (u \rightarrow v) \wedge (\forall u_v \in V  ,  (v \rightarrow u_v)  \Rightarrow (u \rightarrow u_v)  ).$$

  \item[Existencia del súbdito:]
    Para demostrar que cada torneo es un súbdito, basta notar que un rey en un torneo $G$ es un súbdito en el torneo $\widehat G$ que se obtiene cambiando la orientación de cada arista en $G$, así, por su significado simétrico, como cada torneo tiene un rey, cada torneo tiene un súbdito.\\

\item[\bf{Pregunta 2}] Presenta un algoritmo para encontrar un rey y analiza su complejidad y correctez.

\item[Algoritmo]
  El algoritmo es el sigiente:
  Recorrase el torneo buscando el elemento que tiene mayor exvecindad.

\begin{lstlisting}[frame=single] 
  maximo = none
  rey = none
  for {v | v in V}:
     if |v| > maximo:
       maximo = |v|
       rey = v
  return maximo
\end{lstlisting}

Este algoritmo busca exhaustivamente sobre los vértices, por lo que necesariamente encuentra aquel de máxima exvecindad.\\
Asumiendo que la representación de $G$ es la gráfica de adyacencia, el algoritmo es del orden de $O(n + m)$, pues cada vértice se recorre una vez y ya en el y solo se toma el tamaño de la lista de adyacencia, lo cual es en el caso de una lista ligada $O(m)$.

\item[\bf{Pregunta 3}] Demuestra que todo torneo tiene un camino dirigido Hamiltoniano.
\item[Por inducción:]
  denotaremos $x \rightarrow  y, si (x,y)\in E$\\
  El caso base es la gráfica de 1 vértice, el camino es el mismo vértice.\\
  Supongamos que es cierto que todo torneo de tamaño $n$ tiene un camino Hamiltoniano.\\
  Sea $T=(V,E)$ un torneo donde $|V|=n+1$\\
  Considere $v \in V$ y el torneo $T_{-v}$ inducido al eliminar $v$ y las aristas incidentes a $v$ de $T$, sabemos por hipotesis de inducción que en $T_{-v}$ hay un camino Hamiltoniano.\\
    Sea $H_{n}=(v_1,v_2,...,v_n)$ tal camino.
      Si $v \rightarrow v_{1}$, $(v,v_1,v_2,...,v_n)$ es un camino Hamiltoniano en $T_{-v}$.\\
      En caso contrario $v_1 \rightarrow v$, sea $v_i$ el primer elemento en $H_{n}$ tal que $v \rightarrow v_i$, ahi tenemos que $(v_1,v_2,...,v_{i-1},v,v_i,...,v_n)$ es un camino Hamiltoniano en $G$, pues sabemos que $j<i \Rightarrow v_{j} \rightarrow v$.
      Y si no hay un primer elemento, es decir $\forall u \in H, u \rightarrow v$, entonces $(v_1,v_2,...,v_n,v)$ es un camino Hamiltoniano en $T \blacksquare$.\\

\item[\bf{Pregunta 4}] Prueba que en un torneo aleatorio de n vértices, la probabilidad de que todo vértice sea un rey y un subdito tiende a 1 conforme n tiende a infinito.

\item[Respuesta] 
Sea $T=(V,E)$ un torneo con $|V|=n$.\\ 
denotaremos $x \rightarrow  y, si (x,y)\in E$\\

Diremos que un vértice $x_2$ resiste a $x_1$ si $x_2 \rightarrow x_1$ y $\forall v \in V x_1 \rightarrow  v \Rightarrow v \rightarrow x_2$

Calculemos la probabilidad $\psi(x_1,x_2)$ de que $x_2$ resista a $x_1$, dado que $p(x_1 \rightarrow x_2)=1/2$ y que para cada $u \in V$  $ p(v\rightarrow x_2 \vee x_1\rightarrow v )=1/4$ la probabilidad de que esto no pase para los $n-2$ vértices restantes es $\psi(x_1,x_2)=(1/2)\times(3/4)^{n-2}$.
y $lim_{n\to\infty} \psi(x_1,x_2)=(1/2)\times(3/4)^{n-2} = 0 \blacksquare$
El caso de subdito, es demostrado dado la relación simetrica de la propiedad de ser rey y ser súbdito.


\item[\bf{Pregunta 5}]Sean $T_{1}$ y $T_{2}$ dos torneos sobre el mismo conjunto de vértices. Demuestra la existencia de un vértice desde el cual se puede llegar a todos los demás en un arco de $T_{1}$ , o un arco de $T_{2}$ , o un arco de $T_{1}$ seguido de uno en $T_{2}$.

Llamemos rey doble, al vértice que cumple la condicion anterior\\

\item[Inducción] Para el caso de los torneos de Tamaño 1, el único vértice es el rey, tanto en $T_1$ como en $T_2$ por vacuidad en el número de aristas.

  denotemos $x \Rightarrow y , x,y \in V$ si $$(x \rightarrow y \in T_1) \vee (x \rightarrow y \in T_2) \vee (\exists z \in V | x \rightarrow z \in T_1 \wedge z \rightarrow y \in T_2)$$ 

  Supongamos que es cierto que para todo torneo de tamaño $n$ existe un rey doble
  por demostrar, existe un rey noble en todo par de torneos sobre el mismo conjunto de n vértices $V$.\\
  sea $j$ un vértice cualquiera en $V$, en las gráfica inducida $V_{-j}$ por los 2 Torneos $T_1$ y $T_2$ al retirar $j$ por hipotesis de inducción hay un rey doble $r$ en $V_{-j}$.\\
Caso 1: $r \rightarrow j $ en $T_1$ o en $T_2$  por lo tanto $r$ es un rey

Caso 2: $j \rightarrow t $ tanto en $T_1$ como en $T_2$, considere $\delta_{T1}(r)$ que 
es la exvecindad de $r$ en $T_1$, si $ \exists v \in V | v \rightarrow j  \in T_2 $ entonces $r$ es rey doble, en caso contrario  $j \rightarrow u $ en $T_2$ a todo 
vértice en $\delta_{T_1}(r)$ pero como $j \rightarrow t $ en $T_1 $  y
$ j \Rightarrow \delta_{T1}(r) $ luego $ j \Rightarrow \delta_{T_2}(r) $ puesto que 
$j \rightarrow t $ en $T1$ y también $ j \Rightarrow \delta_{T_2}(\delta_{T_1}(r))$ 
puesto que $j \rightarrow \delta_{T_1}(r) $ en $T_1$, pero por $r$ ser rey doble 
$r \cup \delta_{T_1}(r) \cup \delta_{T_2}(r) \cup \delta_{T_2}(\delta_{T_1}(r)) = V_{-j}$ $j$ es rey doble en $V \blacksquare$

\item[\bf{Pregunta 6}]El algoritmo de BFS visto en clase utiliza una cola de vértices, y cada vértice tiene asignado un color (gris,blanco,negro), y un estimado de distancia d. El estimado d(v) siempre es mayor o igual a la distancia del origen a v, y al final del algoritmo, d(v) es igual a la distancia del origen a v. Presenta y demuestra las invariantes que satisfacen los vértices en la cola, con respecto a d y a su color.

Por claridad cambiemos (gris,blanco,negro) por (encontrado,visitados y no visitado)
\begin{lstlisting}[frame=single] 
1 def BFS(G,v):
2    create a queue Encontrados
3    create a set Visitados
4    create queue Ordenados
5    enqueue v onto Encontrados
6    while Encontrados is not empty:
7        t <- Encontrados.dequeue()
8        for all vertex u in G.adjacentEdges(t) do
9             if u is not in Ordenados:
10                if u is not in Encontrados:
11                   enqueue u into Encontrados
12            add u into Ordenados
\end{lstlisting}

\item[invariantes]
  \begin{itemize}
 \item En cada iteración de $(6)$ el conjunto Ordenados aumenta en uno su tamaño.\\
   Lo cual es cierto pues $12$ es ejecutado incondicionalmente una vez cada que se pasa por $6$.
   \item Cada elemento entra una unica vez y sale una unica vez de Encontrados.\\
     Esto se debe a que $11$ es condicion de $9$ y de $10$, es decir, que se añade a Encontrados, si no esta en Encontrados o en Ordenados, es decir que se visita por primera vez, de ahi que solo entra una vez a Encontrados
   \end{itemize}


\item[\bf{Pregunta 7}] Adapta el algoritmo del emparejamiento estable para instancias de n hombres y m mujeres en donde n = m. El objetivo es encontrar un emparejamiento estable de cardinalidad máxima, con la definición original de emparejamiento estable (es decir, que no hay alguna pareja inestable). Demuestra que tu adaptación siempre termina y que da un emparejamiento estable de cardinalidad máxima posible.

¿Cuál es el tiempo de ejecución del algoritmo en términos de n y m?
¿Se puede caracterizar el tamaño del emparejamiento estable de máxima
cardinalidad en términos de n y m?
¿Tu adaptacién sigue optimizando individualmente a cada hombre (cada hombre termina con su mejor pareja válida)?

\item[Respuesta] Adaptación: supongamos que $n > m$, lo que hacemos es crear $n-m$ mujeres ficticias de tal forma que cada hombre siempre prefiera a cualquier mujer real sobre cualquier mujer ficticia, en caso de $m > n$ creamos $m-n$ hobres ficticios, de tal forma que cada mujer prefiera a cualquier hombre real sobre cualquier hombre ficticio. en el caso $n = m$ es el visto en clase y alimentamos con estos datos el algoritmo $G-S$.
\item[Tiempo de ejecución] Como lo ``reducimos'' al caso visto en clase cuya complejidad es de $\theta(n^2)$ añadiendo elementos ficticios para hacerlo cuadricular, la complejidad es de  $\theta(max(n,m)^2)$

\item[Optimilidad] Sí, esto se debe a que si un hombre obtiene su mejor pareja válida (real o ficticia), en el caso de que sea ficticia, como el prefiere a cualquier mujer real sobre cualquier ficticia por construcción quiere decir que no habia pareja válida real para el y por lo tanto la mujer que recibe (i.e. ninguna) es la mejor pareja válida, en caso de que le toque pareja real, la conclusión se sigue de $G-S$ directamente.

\item[\bf{Pregunta 8}] Sean P y Q dos programas que ordenan números. P lo hace con mergesort, cuya complejidad es $\theta(n log n)$ y Q con BubbleSort, cuya complejidad es $\theta(n^2)$. Juanito Hacker dice que corrio a P y Q en la misma computadora, dándoles como entrada a ambos una lista de $1,000,000$ de elementos, y que consistentemente  P hizo 1000 veces más operaciones más que Q. ¿Es esto posible?

\item[Respuesta] Esto es perfectamente posible y en el casos de BubbleSort y de MergeSort fácilmente explicable, pues resulta que BubbleSort es $\theta(n^2)$ en el peor de los casos y $\theta(n)$ en el caso ideal, en tanto que MergeSort tiene una complejidad $\theta(n log n)$ en el peor caso y tambien en el caso esperado por lo que existen instancias donde BubbleSort es lineal y Mergesort $\theta(n log n)$ (por ejemplo cuando la entrada esta ordenada o solo tiene un número de transposiciones de elementos consecutivos sobre la lista ordenada), y como $n = 1,000,000 = 1,000^2$ $nlogn$ es $1000$ veces $n$, lo cual coincide perfectamente con los datos de Juanito Hacker, que aqui entre nos, parece ser un hacker de peluche si no entiende la ejecución y complejidad de los programas y lo sorprende en lo más mínimo este resultado.

\item[\bf{Pregunta 9}]Sean P y Q dos programas concretos de complejidades $\theta(n^2)$ y $\theta(n)$ respectivamente que resuelven el mismo problema. Ante una instancia I de tamaño n, P hace 100 veces más operaciones que Q. Si ejecutamos P y Q con otra instancia I de tamaño n, ¿que podemos asegurar respecto al número de operaciones de ambos programas?

\item[Respuesta] Nada absolutamente, $\theta(n^2)$ y $\theta(n)$ son construcciones no acotadas por un tamaño específico, es decir un programa puede tomar cualquier tiempo arbitrario para una instancia y segir siendo $\theta(n^2)$ o $\theta(n)$, estas notaciones solo explican el comportamiento \emph{asintótico}, una instancia especifica, o de hecho un tamaño específico de la entrada no tiene ninguna garantía de ningún tipo.

\item[\bf{Pregunta 8}]Supón que tienes programas concretos que se ejecutan hoy en una computadora a cierta velocidad. Supón que dentro de dos años tendrás una computadora el doble de rápida. ¿Qué tan grandes serán los tamaños de los problemas que podrás resolver en dos años, usando una hora de procesamiento, respecto a los tamaños máximos de los problemas que puedes resolver hoy, usando una hora de procesamiento, si tus complejidades son algunas de las siguientes:\\

$\theta(log(n))$, $\theta(log^2 n)$, $\theta(n)$, $\theta(nlogn)$, $\theta(n^2)$, $\theta(n^2logn)$, $\theta(n^3)$, $\theta(1.1^n )$, $\theta(2^n)$ y $\theta(3^n)$ Por supuesto ignora detalles de arquitectura (tamaño de memoria, jerarquía de memoria, etc.) y concéntrate unicamente en las características de escalabilidad teóricas que te da cada complejidad diferente.

\item[Respuesta] Supongamos para la simplicidad que si aqui decimos que un programa es de $\theta(n^2)$ sea exactamente $C*n^2 \forall |I|=n$ con $C$ una constante arbitraria pero fija, y más aún, supongamos por simplicidad que esa constante $C$ es tal que por haga que para cada problema, pudiera resolver una entrada de tamaño 1000 el dia de hoy.
así teniendo mañana una computadora 2 veces más rápida nos daría:

\begin{tabular}{ | l | l | l | l  |}
  \hline
Orden   & Tamaño hoy  &  Tamaño mañana\\
  \hline
$\theta(log(n))$  & 1,000    & 1,000,000\\
$\theta(log^2(n))$ & 1,000   &    17,484\\
$\theta(n)$       & 1,000    &     2,000\\
$\theta(nlog(n))$   & 1,000  &     1,838\\
$\theta(n^2)$     & 1,000    &     1,412\\
$\theta(n^2log(n))$ & 1,000  &     1,382\\
$\theta(n^3)$     & 1,000    &     1,260\\ 
$\theta(1.1^n )$  & 1,000    &     1,007\\
$\theta(2^n)$     & 1,000    &     1,001\\
$\theta(3^n)$     & 1,000    &   1,000.6\\ 
  \hline  
\end{tabular}

\end{itemize}
\end{document}
